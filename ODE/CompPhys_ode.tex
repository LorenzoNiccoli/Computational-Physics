\documentclass[12pt,a4paper]{article}
\usepackage[utf8]{inputenc}
\usepackage[T1]{fontenc}
\usepackage[english,italian]{babel}
\usepackage[a4paper, margin=2.54cm]{geometry}
\usepackage{amsmath, amssymb}   
\usepackage{setspace}           
\usepackage{parskip}            
\usepackage{array}

\title{ODE}
\author{Lorenzo Niccoli}
\date{}

\begin{document}

\maketitle

\section{Solving Ordinary Differential Equations}
The first set of problems we address are ordinary differntial equation of the form: 

\begin{equation}
    \frac{dy}{dt} = f(y,t) \qquad y(t_0) = y_0
\end{equation}
where \(y\) is the unknown function, \(t\) is the independent variable, and \(f(y,t)\) 
is a given function. The initial condition \(y(t_0) = y_0\) specifies the value of the 
solution at time \(t_0\).\\

This is a quite general formulation that includes many problems of interest in science and engineering. 
This equation models, for example, simple physical problems such asradiocative decay. 

\renewcommand{\arraystretch}{2}

\begin{table}[h!]
\centering
\caption{Some physical phenomena modeled by ordinary differential equations (ODEs).}
\begin{tabular}{|>{\centering\arraybackslash}m{6cm}|
                    >{\centering\arraybackslash}m{8cm}|}
\hline
\textbf{Physical phenomena} & \textbf{ODEs} \\
\hline
Radioactive decay           & $\frac{dN}{dt} = -\lambda N$ \\
\hline
Exponential growth          & $\frac{dy}{dt} = k y$ \\
\hline
Newton cooling law          & $\frac{dT}{dt} = -k(T - T_\text{amb})$ \\
\hline
Charge/Discharge plate (RC) & $\frac{dV}{dt} = \frac{1}{RC}(V_0 - V)$ \\
\hline
Logistic growth             & $\frac{dP}{dt} = rP(1-P/K)$ \\
\hline
First order kinetics        & $\frac{d[A]}{dt} = -k[A]$ \\
\hline
\end{tabular}
\end{table}


\subsection{Euler algorithm}
The simplest numerical method for solving ordinary differential equations is the Euler method. 
It is based on the idea of approximating the solution by a series of small steps, using a Taylor expantion up 
to the first order.

\begin{equation}
    y(t_0 + \Delta t) \approx y(t_0) + \Delta t \frac{dy}{dt} = y_0 + \Delta t f(y_0, t_0) + O(\Delta t^2)
\end{equation}

Iterating this equation and introducing a discrete time variable \(t_n = t_0 + n \Delta t\), we obtain the Euler algorithm:

\begin{equation}
    y_{n+1} = y_n + \Delta t f(y_n, t_n) + O(\Delta t^2)
\end{equation}

In contrast to the analytical solution which was easy for the simple example given above, the Euler method retains
its simplicity nomatter which differential equations it is applied to.

\end{document}


